% Für Seitenformatierung

\documentclass[DIV=15]{scrartcl}

% Zeilenumbrüche

\parindent 0pt
\parskip 6pt

% Für deutsche Buchstaben und Synthax

\usepackage[ngerman]{babel}

% Für Auflistung mit speziellen Aufzählungszeichen

\usepackage{paralist}

% zB für \del, \dif und andere Mathebefehle

\usepackage{amsmath}
\usepackage{commath}
\usepackage{amssymb}

% für nicht kursive griechische Buchstaben

\usepackage{txfonts}

% Für \SIunit[]{} und \num in deutschem Stil

\usepackage[output-decimal-marker={,}]{siunitx}
\usepackage[utf8]{inputenc}

% Für \sfrac{}{}, also inline-frac

\usepackage{xfrac}

% Für Einbinden von pdf-Grafiken

\usepackage{graphicx}

% Umfließen von Bildern

\usepackage{floatflt}

% Für Links nach außen und innerhalb des Dokumentes

\usepackage{hyperref}

% Für weitere Farben

\usepackage{color}

% Für Streichen von z.B. $\rightarrow$

\usepackage{centernot}

% Für Befehl \cancel{}

\usepackage{cancel}

% Für Layout von Links

\hypersetup{
	citecolor=black,
	colorlinks=true,
	linkcolor=black,
	urlcolor=blue,
}

% Verschiedene Mathematik-Hilfen

\newcommand \e[1]{\cdot10^{#1}}
\newcommand\p{\partial}

\newcommand\half{\frac 12}
\newcommand\shalf{\sfrac12}

\newcommand\skp[2]{\left\langle#1,#2\right\rangle}
\newcommand\mw[1]{\left\langle#1\right\rangle}
\renewcommand \exp[1]{\mathrm e^{#1}}

% Nabla und Kombinationen von Nabla

\renewcommand\div[1]{\skp{\nabla}{#1}}
\newcommand\rot{\nabla\times}
\newcommand\grad[1]{\nabla#1}
\newcommand\laplace{\triangle}
\newcommand\dalambert{\mathop{{}\Box}\nolimits}

%Für komplexe Zahlen

\renewcommand \i{\mathrm i}
\renewcommand{\Im}{\mathop{{}\mathrm{Im}}\nolimits}
\renewcommand{\Re}{\mathop{{}\mathrm{Re}}\nolimits}

%Für Bra-Ket-Notation

\newcommand\bra[1]{\left\langle#1\right|}
\newcommand\ket[1]{\left|#1\right\rangle}
\newcommand\braket[2]{\left\langle#1\left.\vphantom{#1 #2}\right|#2\right\rangle}
\newcommand\braopket[3]{\left\langle#1\left.\vphantom{#1 #2 #3}\right|#2\left.\vphantom{#1 #2 #3}\right|#3\right\rangle}


\setcounter{section}{1}
\renewcommand\thesection{H\,1.\arabic{section}}
\renewcommand\thesubsection{\thesection.\alph{subsection}}

\title{physik521: Übungsblatt 01}
\author{Lino Lemmer \and Martin Ueding \\ \small{\texttt{mu@martin-ueding.de}}}

\begin{document}
\maketitle
\section{Maxwell Relationen und Ableitungsregeln}


\section{Wahrscheinlichkeitstheorie}

\subsection{Zufallsvariable}

\[
    X = \text{Anzahl der geraden Augenzahlen}
\]

\subsection{Wahrscheinlichkeitsverteilung}

Die Wahrscheinlichkeitsverteilung von $X$ ist
\begin{align*}
    P_X(0) &= \half\cdot\half = \frac 14\\
    P_X(1) &= \half\cdot\half + \half\cdot\half = \half\\
    P_X(2) &= \half\cdot\half = \frac 14\\
    \sum P_X &= \frac 14 + \half + \frac 14 = 1
\end{align*}

\subsection{Mittelwert}

Der Mittelwert zur Zufallsvariable $X$ ist
\[
    \mw{X} = 0\cdot\frac14 + 1\cdot\half + 2\cdot\frac14 = 1
\]

\subsection{Schwankungsquadrat}

Das Schwankungsquadrat von $X$ ist
\begin{align*}
    \Delta X &= \sqrt{\mw{\del{X - 1}^2}}\\
             &= \sqrt{\frac 14 \cdot 1 + \half \cdot 0 + \frac 14 \cdot 1}\\
             &= \frac1{\sqrt{2}}
\end{align*}

\subsection{Korrelationsfunktion}

\subsubsection{Würfel}

Ich berechne zunächst die Mittelwerte für den Würfel
\begin{align*}
    \mw{X_1^1} &= \mw{X_2^1} \\
               &= \frac 16 \del{1+2+3+4+5+6} \\
               &= \num{3.5} \\
    \intertext{Die Korrelationsfunktion ist}
    K_{12} &= \mw{ \del{ X_1^1 - \mw{X_1^1} } \del{ X_2^1 - \mw{X_2^1}}} \\
           &= \mw{ \del{ X_1^1 - \num{3.5}}\del{ X_2^1 - \num{3.5}}}
    \intertext{Da genauso viel negative Abweichungen von den Mittelwerten wie positive Auftauchen, folgt}
    &= 0
\end{align*}
Die Zufallsgrößen sind also unkorreliert.

\subsubsection{Fußball-Physiker}

Nun für den Fußball-Physiker
\begin{align*}
    \mw{X_1^2} &= \\
    \mw{X_2^2} &=
\end{align*}

\subsection{Bedingte Wahrscheinlichkeiten}

Ich verwende folgende Bezeichnungen:
\begin{align*}
    + &= \text{positive Probe} \\
    - &= \text{negative Probe} \\
    \text{d} &= \text{gedopt} \\
    \overline{\text{d}} &= \text{nicht gedopt}
    \intertext{Für die Wahrscheinlichkeit eines Ereignisses $A$ unter der Bedingung $B$ verwende ich folgende Notation:}
    &P_B\del{A}
\end{align*}

\subsubsection{Positive Dopingprobe}

\begin{align*}
    P\del{+} &= P\del{\text{d}} \cdot P_\text{d}\del{+} + P\del{\overline{\text{d}}} \cdot P_{\overline{\text{d}}}\del{+} \\
             &= \num{0.2} \cdot \num{0.99} + \num{0.8} \cdot \num{0.05} \\
             &= \num{0.238}
\end{align*}

\subsubsection{Negative Probe trotz Doping}

\begin{align*}
    P_\text{d}\del{-} &= \num{0.01}
    \intertext{Ich vermute, dass die Frage auf die Wahrscheinlichkeit für einen falsch negativen Test abzielt. Diese ist}
    P\del{\text{d}\wedge-} &= P\del{\text{d}} \cdot P_\text{d}\del{-} \\
                           &= \num{0.2} \cdot \num{0.01} \\
                           &= \num{0.002}
\end{align*}

\end{document}
