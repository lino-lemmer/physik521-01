% Für Seitenformatierung

\documentclass[DIV=15]{scrartcl}

% Zeilenumbrüche

\parindent 0pt
\parskip 6pt

% Für deutsche Buchstaben und Synthax

\usepackage[ngerman]{babel}

% Für Auflistung mit speziellen Aufzählungszeichen

\usepackage{paralist}

% zB für \del, \dif und andere Mathebefehle

\usepackage{amsmath}
\usepackage{commath}
\usepackage{amssymb}

% für nicht kursive griechische Buchstaben

\usepackage{txfonts}

% Für \SIunit[]{} und \num in deutschem Stil

\usepackage[output-decimal-marker={,}]{siunitx}
\usepackage[utf8]{inputenc}

% Für \sfrac{}{}, also inline-frac

\usepackage{xfrac}

% Für Einbinden von pdf-Grafiken

\usepackage{graphicx}

% Umfließen von Bildern

\usepackage{floatflt}

% Für Links nach außen und innerhalb des Dokumentes

\usepackage{hyperref}

% Für weitere Farben

\usepackage{color}

% Für Streichen von z.B. $\rightarrow$

\usepackage{centernot}

% Für Befehl \cancel{}

\usepackage{cancel}

% Für Layout von Links

\hypersetup{
	citecolor=black,
	colorlinks=true,
	linkcolor=black,
	urlcolor=blue,
}

% Verschiedene Mathematik-Hilfen

\newcommand \e[1]{\cdot10^{#1}}
\newcommand\p{\partial}

\newcommand\half{\frac 12}
\newcommand\shalf{\sfrac12}

\newcommand\skp[2]{\left\langle#1,#2\right\rangle}
\newcommand\mw[1]{\left\langle#1\right\rangle}
\renewcommand \exp[1]{\mathrm e^{#1}}

% Nabla und Kombinationen von Nabla

\renewcommand\div[1]{\skp{\nabla}{#1}}
\newcommand\rot{\nabla\times}
\newcommand\grad[1]{\nabla#1}
\newcommand\laplace{\triangle}
\newcommand\dalambert{\mathop{{}\Box}\nolimits}

%Für komplexe Zahlen

\renewcommand \i{\mathrm i}
\renewcommand{\Im}{\mathop{{}\mathrm{Im}}\nolimits}
\renewcommand{\Re}{\mathop{{}\mathrm{Re}}\nolimits}

%Für Bra-Ket-Notation

\newcommand\bra[1]{\left\langle#1\right|}
\newcommand\ket[1]{\left|#1\right\rangle}
\newcommand\braket[2]{\left\langle#1\left.\vphantom{#1 #2}\right|#2\right\rangle}
\newcommand\braopket[3]{\left\langle#1\left.\vphantom{#1 #2 #3}\right|#2\left.\vphantom{#1 #2 #3}\right|#3\right\rangle}


\setcounter{section}{1}
\renewcommand\thesection{H\,1.\arabic{section}}

\title{physik521: Übungsblatt 01}
\author{Lino Lemmer}

\begin{document}
\maketitle
\section{Maxwell Relationen und Ableitungsregeln}

\begin{enumerate}[(a)]
    \item
    \item
    \item
\end{enumerate}

\section{Wahrscheinlichkeitstheorie}

\begin{enumerate}[(a)]
    \item
        \[
            X = \text{Anzahl der geraden Augenzahlen}
        \]
    \item
        Die Wahrscheinlichkeitsverteilung von $X$ ist
        \begin{align*}
            P_X(0) &= \half\cdot\half = \frac 14\\
            P_X(1) &= \half\cdot\half + \half\cdot\half = \half\\
            P_X(2) &= \half\cdot\half = \frac 14\\
            \sum P_X &= \frac 14 + \half + \frac 14 = 1
        \end{align*}
    \item
        Der Mittelwert zur Zufallsvariable $X$ ist
        \[
            \mw{X} = 0\cdot\frac14 + 1\cdot\half + 2\cdot\frac14 = 1
        \]
    \item
        Das Schwankungsquadrat von $X$ ist
        \begin{align*}
            \Delta X &= \sqrt{\mw{\del{X - 1}^2}}\\
                       &= \sqrt{\frac 14 \cdot 1 + \half \cdot 0 + \frac 14 \cdot 1}\\
                       &= \frac1{\sqrt{2}}
        \end{align*}
    \item
        Ich berechne zunächst die Mittelwerte für den Würfel
        \begin{align*}
            \mw{X_1^1} &= \mw{X_2^1} \\
                       &= \frac 16 \del{1+2+3+4+5+6} \\
                       &= \num{3.5} \\
            \intertext{Die Korrelationsfunktion ist}
            K_{12} &= \mw{ \del{ X_1^1 - \mw{X_1^1} } \del{ X_2^1 - \mw{X_2^1}}} \\
                     &= \mw{ \del{ X_1^1 - \num{3.5}}\del{ X_2^1 - \num{3.5}}}
            \intertext{Da genauso viel negative Abweichungen von den Mittelwerten wie positive Auftauchen, folgt}
                    &= 0
            \intertext{Die Zufallsgrößen sind also unkorreliert.
            Nun für den Fußball-Physiker}
            \mw{X_1^2} &= \\
            \mw{X_2^2} &= \\
        \end{align*}
    \item
    \item
\end{enumerate}
\end{document}
